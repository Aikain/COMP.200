Ansible on yksi monista eri vaihtoehdoista toteuttaa infrastruktuuri koodina.
Tässä kappaleessa esitellään lyhyesti mitä infrastruktuuri koodina on ja
mitä eri vaihtoehtoja on. Tämän lisäksi käydään läpi Ansiblen perusteita,
joiden tietäminen on välttämätöntä myöhempien lukujen kannalta.
TODO: lähteet

\section{Infrastruktuuri koodina}

Infratruktuuri koodina on nimensä mukaisesti sitä, että infrastruktuuri
muodostetaan koodin pohjalta. Lähtökohtaisesti ympäristö on tällöin
luotu automaattisesti ja määritellyllä tavalla, joka on mahdollista
toteuttaa uudelleen samanlaisena. Infrastruktuurin toteuttamienn koodin
avulla mahdollistaa koko infrastruktuurin dokumennoin toteutuksen ohessa
eikä jätä ympäristöön liittyviä asioita tekijän muistin varaan.
TODO: lähteet

Tunnettuja IaC-ratkaisuja ovat mm. Ansible, Puppet, Chef, SaltStack ja
Terraform. Idea eri ratkaisuissa on pääpiirteittä sama. Suurimpia eroja
tuvoat kieli, jolla ratkaisu on toteutettu sekä ajamisessa käytettävä
metodi. Osa ratkaisuista, kuten cef ja puppet, suosivat veto-metodia
kun puolestaan Ansible ja Terraform työntö-metodia. Veto-metodissa
käyttäjä suorittaa ohjelman kohdepalvelimella, joka hakee tarvittavat
määritykset hallintapalvelimelta. Työntö-metodissa puolestaan ohjelma
ottaa yhteyden palvelimille.
TODO: lähteet

\section{Ansiblen asennus ja käyttö}

Versiosta 2.10 alkaen Ansible on jakautunut kahteen eri pakettiin:
ansible ja ansible-core. Varsinainen Ansiblen ydin löytyy ansible-core
-paketista. Ansible paketti puolestaan sisältää yhteisön tarjoaman
valmiin kattauksen moduuleja, joita voidaan käyttää eri tehtävissä.
\parencite{AnsibleDocs}

Ansible tarvitsee toimiakseen pythonin, joten käyttöjärjestelmästä
löytyy todennäköisesti myös pip, jolla Ansiblen asennus onnistuu
vaivattomasti:
TODO: lähteet

\begin{lstlisting}[
    basicstyle=\small,
    label={lst:install-ansible},
    language=bash
]
    $ pip3 install ansible
\end{lstlisting}

Tämän jälkeen voidaan toiminnallisuus testata esim.
TODO: lähteet

\begin{lstlisting}[
    basicstyle=\small,
    label={lst:ansible-ping},
    language=bash
]
    $ ansible localhost -m ping
\end{lstlisting}

\subsection{Inventaario}

Ansiblella hallittavat palvelimet luetteloidaan inventaariossa.
Yksinkertaisimmillaan inventaario sisältää listan palvelimien nimistä:
TODO: lähteet

\begin{lstlisting}[
    basicstyle=\small,
    label={lst:ansible-inventory},
]
    mail.example.com

    [webservers]
    foo.example.com
    bar.example.com

    [dbservers]
    one.example.com
    two.example.com
    three.example.com
\end{lstlisting}

Inventaariot voivat myös olla dynaamisia. Tällöin lista palvelimista
voidaan hakea esimerkiksi palveluntarjoajan APIsta, kuten Hetzner Cloudin
palvelimet saadaan haettua käytätmällä hetzner.hcloud.hcloud-inventaariota.
TODO: lähteet

\subsection{Playbookit ja tehtävät}

Ansiblessa suoritaan yksittäisiä pieniä tehtäviä, joita peräkkäin ajamalla
saadaan toteutettua kokonaisuuksia. Tehtävät käyttävät jotain asennettua
moduulia. Esimerkissä \ref{lst:ansible-apt-figlet} käytetään apt-moduulia,
jolla saadaan asennettua määritelty paketti. Esimerkin mukaisesti tehtäville
annetaan kuvaava nimi, joka tulee näkymään myöhemmin ulostulossa kun
tehtäviä ajetaan.
TODO: lähteet

\lstinputlisting[
    basicstyle=\small,
    caption={Figlet-ohjelman asentaminen Ansiblen apt-moduulilla},
    label={lst:ansible-apt-figlet},
]{code/install-figlet.yaml}

Playbookit puolestaan määrittelevät mitä tehtäviä ajetaan minnekkin.
TODO: lähteet

\lstinputlisting[
    basicstyle=\small,
    caption={Esimerkki playbookista},
    label={lst:ansible-playbook},
]{code/playbook-example.yaml}

Ylläolevassa esimerkissä on määritelty ajettavaksi hcloud-ryhmälle, joka
on saadaan haettua inventaariosta, ensimmäisenä yksittäinen tehtävä ja
tämän jälkeen kaksi roolia.
TODO: lähteet

Playbook saadaan ajettua ansible-playbook -komennolla:
TODO: lähteet

\begin{lstlisting}[
    basicstyle=\small,
    label={lst:run-ansible-playbook},
    language=bash
]
    $ ansible-playbook -i inventory.ini playbook.yml
\end{lstlisting}

\subsection{Roolit, sapluunat ja käsittelijät}

Yksittäisten tehtävien kirjoittamisen sisään on mahdollista tehdä rooleja,
jotka sisältävän useita tehtäviä ja muita niiden tarvitsemia asioita, kuten
sapluunoita (template) tai käsittelöitä (handlers). Samaa roolia voidaan
käyttää myös useissa eri playbookeissa, jolloin rooleilla saadaan muodostettua
pieniä kokonaisuuksia, joita yhdistelemällä saadaan muodostettua koko
kokonaisuus. Roolin kansio rakenne on alla olevan esimerkin mukainen:
TODO: lähteet

\begin{lstlisting}[
    basicstyle=\small,
    label={lst:ansible-role-structure},
]
    roles/
        example-role/
            tasks/
            handlers/
            library/
            files/
            templates/
            vars/
            defaults/
            meta/
\end{lstlisting}

Yksinkertaisimmillaan rooli sisältää vain tasks-kansion, jossa on määritelty
main.yml-tiedosto, joka sisältää kaikki roolin ajettavat tehtävät. Rooli
voi siis käytännössä olla vain yksi tehtävä, kuten aiemmin esitelty esimerkki
yksittäisestä tehtävästä \ref{lst:ansible-apt-figlet}.
TODO: lähteet

Roolit voivat sisältää myös muita hyödyllisiä työkaluja, kuten sapluunoita tai
käsittelijäitä. Sapluunoiden avulla voidaan rakentaa esimerkiksi ohjelma
asetustiedostolle pohja, johon täytetään tiedot palvelinkohtaisesti. Sapluuna
muodostetaan jinja2-kieltä käyttämällä.
TODO: lähteet

\lstinputlisting[
    basicstyle=\small,
    caption={Esimerkki käyttäjän ssh-avaimien listaamisesta (authorized\_keys)},
    label={lst:template-example},
]{code/template-example.j2}

Jinja2-kieli tukee datan yksinkertaista käsittelyä, kuten ylläolevan esimerkin
mukaisesti listojen läpi käyntiä. Sapluunoja käytetään template-moduulin avulla,
josta on esimerkki käsittelijä-esimerkin yhteydessä \ref{lst:handler-task-example}.
TODO: lähteet

Ohjelmien asennus voi vaatia useita pieniä tehtäviä, jotka kaikki ovat
tarpeellisia, jotta ohjelma toimii oikein. Tällöin ohjelman käynistämistä ei
voida välttämättä toteuttaa ennen kuin kaikki tehtävät ovat suoritettu.
Tällöin on mahdollista merkitävä tehtäville käsittelijä. Käsittelijä ajetaan
vasta kun kaikki tehtävät ovat suoritettu. Yleinen käyttö käsittelijöille
on nimenomaan asetustiedostojen yhteydessä. Käsittelijä ajetaan vain jos
tehtävä, johon se on liitetty, tekee muutoksia. Tällöin voidaan esimerkiksi
käynnistää ohjelma uusiksi mikäli asetustiedostossa on muutoksia ja ohjelma
tarvitsee uudelleenkäynnistyksen ladatakseen uudet muutokset.
TODO: lähteet

Käsittelijä laitetaan handlers-kansioon main.yml-tiedostoon:
TODO: lähteet

\lstinputlisting[
basicstyle=\small,
caption={Esimerkki käsittelijän toteuttamisesta},
label={lst:handler-example},
]{code/handler-example.yaml}

jonka jälkeen käsittelijää voidaan käyttää tehtävän yhteydessä notify-avainsanan
avulla:
TODO: lähteet

\lstinputlisting[
    basicstyle=\small,
    caption={Esimerkki käsittelijän toteuttamisesta},
    label={lst:handler-task-example},
]{code/handler-task-example.yaml}

\subsection{Muuttujat ja holvi}

Ansiblessa tietoa säilytään muuttujiin. Muuttujia on mahdollista määrittää monella
eri tavalla. Muuttujat voivat olla määritelty esimerkiksi roolissa, inventorissa
tai playbookissa ja ne voivat kohdistua tiettyyn palvelimeen tai palvelimien
muodostomaan ryhmään.
TODO: lähteet

Muuttujiin saatetaan kuitenkin haluta määritellä salasanoja tai muida arkaluontoisia
tietoja. Jotta arkaluontoiset tiedot saadaan salattua, voidaan käyttään Ansiblen
tarjoamaa holvia.
TODO: lähteet

\begin{lstlisting}[
    basicstyle=\small,
    label={lst:ansible-vault},
    language=bash
]
    $ pwgen -s 77 1 | tr -d '\n' | ansible-vault encrypt_string \
        --vault-password-file ansible-vault.secret \
        --encrypt-vault-id default
    Reading plaintext input from stdin. (ctrl-d to end input, twice if
    your content does not already have a newline)

    !vault |
              $ANSIBLE_VAULT;1.1;AES256
              3538646364656536356461313931323831356638643831396135623836
              3563303363643561613032376639646534303366633533333539626535
              3938393536360a63376136303037303037613936316532656561366462
              3335363136313533616331646132633433346330303865363332636135
              346436313331633664396532620a323439626639376666333634333536
              3461313664623036613130363931616231303631383130396261386634
              3134393439326538386665633436306562663330366535666365633232
              3039646361333636333330323035313334396134363232636461333033
              3566373664313361356135616330363838653832613536396639383339
              6164366463343736623635633035393962303933373662666237636633
    Encryption successful

\end{lstlisting}

Ylläolevassa esimerkissä generoidaan satunnainen 77-merkkinen salasana, joka
sen jälkeen välitetään ansible-vault -komennolle, joka salaa generoidun
salasanan ansible-vault.secret -tiedoston avulla. Tämän jälkeen salattua tietoa
voidaan käyttää muuttujien sisältönä:
TODO: lähteet

\lstinputlisting[
    basicstyle=\small,
    caption={Muuttujien määrittely},
    label={lst:var-example},
]{code/var-example.yaml}

Kunhan vain Ansible pystyy saamaan tiedoston, jonka avulla salaus toteutettiin,
pystyy Ansible purkamaan salaisuuden ajon yhteydessä. Nyt on vain huolehdisttava
salaamisessa käytettävän tiedoston välittömisestä kaikille sitä tarvitseville ja
kaikki muut salaisuudet voivat olla versionhallinnassa.
TODO: lähteet
