Tässä työssä perehdyttiin useissa tutkimuksissa havattuihin mahdollisiin
virheisiin silloin kun toteutetaan infrastruktuuri koodina. Työssä tarkasteltiin
nimenomaan Ansiblen näkökulmasta havaittuja virheitä ja esiteltiin ratkaisuja
niihin. Merkittävimmät virheet tulivat ilmi tietoturvahajuja toteuttaneista
tutkimuksista. Muut esitellyt virheet ovat lähtöisin Ansiblelle määritellyistä
huonoista käytännöistä.

Tietoturvahajuissa useita kertoja esiintyneitä virheitä olivat kovakoodatut
salaisuudet ja eheystarkistuksen puuttuminen. Huonoihin käytöntöitin liittyen
esiintyneitä virheitä olivat puolestaan riittämätön modulaarisuus ja
idempotenssin rikkominen. Kovakoodattuihin salaisuuksiin ratkaisu löytyy
Ansiblen holvi-toiminnallisuudesta. Eheystarkistuksille puolestaan tuki
löytyy suoraan moduuleihin sisäänrakennettuna ja vaatii vain sen käyttämisen.
Huonot käytännöt ovat puolestaan riippuvat tehdyistä ratkaisuista. Asioita
voidaan tehdä usealla tavalla, joista tietyissä on ongelmia, joita tulisi välttää.

Tietoturvahajuja ja joitakin huonoja käytänteitä voidaan havaita
automaattisilla työkaluilla. Tietoturvahajuihin keskittyneissä tutkimussa
käytettiin tutkielmien yhteydessä tehtyjä työkaluja. Kyseiset työkalut eivät
kuitenkaan olleet yksinkertaisesti saatavilla, joten tutkielmassa esiteltiin
Will Thamesin luoma työkalu, Ansible Lint, joka on nykyisin Ansiblen yhteisötiimin
jatkokehittämä työkalu ja yleisesti tunnettu hyvänä ratkaisuna Ansiblen kanssa
käytettäväksi.

Vaikka yleisesti IaC:ta pidetään hyvänä ratkaisuna infrastruktuurien
rakentamisessa, on hyvä huomioida, että keinot, joilla pystytään tekemään
pienellä vaivalla paljon asioita, mahdollistavat myös pienillä virheillä
suuren määrän virheitä kun yksittäinen tehtävä voidaan toteuttaa sadoille
palvelimille. On hyvä tietää joistakin yleisimmistä virheistä ja mahdollisista
keinoista havaita niitä. Tänä päivänä keinoja on monia erilaisten ratkaisujen
kehittyessä ja eri ympäristöjen rakentaessa tukea yhä useammille työkaluille.
