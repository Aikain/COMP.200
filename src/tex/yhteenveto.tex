Ohjeilla pyritään mahdollisimman selkeään ja täsmälliseen tekstiin, joka on tärkeää kaikissa kirjallisissa raporteissa. Tämän dokumenttipohjan ja vastaavan Word-pohjan avulla töillä on yhtenäinen ja selkeä ulkoasu.

Jokaisella kirjoituksella ja esityksellä pitää olla yhteenveto. Tätä asiaa korostetaan lisäämällä sellainen tähänkin pohjaan, vaikkakin lyhyenä ja hieman keinotekoisesti. Tiivis yhteenvetotaulukko voi auttaa kertaamaan tärkeimmät kohdat.

Lopuksi vielä mainintoja tästä pohjasta. Valmiin dokumentin tuottamiseksi se on käännettävä pdf\LaTeX{}-ohjelmalla. Tämä vaihtoehto on varsin helposti löydettävissä valmiissa \LaTeX{}-editoreissa, ja komentoriviä käyttäessä riittää kirjoittaa komennoksi \texttt{pdflatex}. Viitteiden ja lähdeluettelon luomiseksi käytetään biber-nimistä ohjelmaa, joka löytyy samaan tapaan. Lyhenne- ja symboliluettelon latomiseksi täytyy ajaa makeindex-niminen ohjelma. Jos vaikuttaa siltä, että sisällysluettelo tai ristiinviittaukset (\verbcommand{ref}) eivät näy oikein, kokeile ajaa pdf\LaTeX{} uudestaan. Jos lopultakin kääntäjä antaa virheraportteja, varmista ensin, että \TeX{}-asennuksesi on ajan tasalla.

Pohja on kirjoitettu Overleaf-ympäristössä, ja kirjoittaja suositteleekin lämpimästi sen version 2 käyttöä opinnäytteiden kirjoittamisessa. Helpoin keino päästä käsiksi dokumenttipohjaan on pyytää kopiointilinkkiä tähän projektiin työn ohjaajalta tai pohjan ylläpitäjältä. Overleafin käyttö vaatii kuitenkin käyttäjätilin ja verkkoyhteyden.

Toivon mukaan ajantasainen versio löytyy myös yliopiston intrasta. Pohja on testattu ja todettu toimivaksi Windows-käyttöjärjestelmän Mik\TeX-ympäristössä ja Unix-järjestelmien täydessä \TeX{} Live -ympäristössä. Näistä ensimmäinen asentaa automaattisesti mahdollisesti puuttuvia paketteja, mutta jälkimmäisen kanssa voi joutua etsimään ja asentamaan itse kokonaisia paketteja tai niiden päivitysversioita.