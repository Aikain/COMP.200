Tutkimuksissa \parencite{RahmanAkond2019TSSS} ja \parencite{RahmanAkond2021SSiA} käytettiin
tutkimuksia varten tehtyjä työkaluja, jotka oli empiirisesti validoitu toimiviksi.
Kyseiset työkalut eivät yleisesti ole tiedossa tai käytettävissä. Tässä luvussa on
kuitenkin esitelty yksi työkalu, Ansible Lint, joka edes auttaa havaitsemaan vastaavia
virheitä ja johon on mahdollista rakentaa itse lisätarkistuksia \parencite{SestoVincent2020ATaV}.

\section{Ansible Lint}

Ansible Lint on työkalu, joka on suunniteltu tarkistamaan noudatetaanko parhaita
käytäntöjä ja sen myötä edesauttamaan käytäntöjen parantamista \parencite{GithubAnsibleLint}.
Työkalua ei ole nimenomaan suunniteltu huomaamaan turvallisuuspuutteita, mutta parhaat
käytännöt sisältävät myös asioita, jotka ovat turvallisuuden kannalta oleellisia.

\subsection{Manuaalinen ajaminen}

Ansible vaatii ajamiseen tarkoitetulla palvelimella Python 3.8:n \parencite{AnsibleDocs}.
Tämän myötä koneelle on todennäköisesti asennettu myös pip. Työkalun asentaminen
onnistuukin tällöin helposti pipin avulla \parencite{AnsibleLintReadTheDocs}:

\begin{lstlisting}[
    basicstyle=\small,
    label={lst:install-ansible-lint},
    language=bash
]
    $ pip3 install ansible-lint
\end{lstlisting}

Työkalu ei tarvitse lisämäärityksiä vaan voi tunnistaa automaattisesti tarvittavat
tiedostot Ansible-projektista \parencite{AnsibleLintReadTheDocs}:

\begin{lstlisting}[
    basicstyle=\small,
    label={lst:run-ansible-lint},
    language=bash
]
    $ ansible-lint
\end{lstlisting}

Komennon antama ulostulo voi näyttää esimerkin \ref{lst:Ansible Lint-simple-example} mukaiselta.

\begin{lstlisting}[
    basicstyle=\small,
    label={lst:ansible-lint-simple-example},
    breaklines=true,
]
    no-handler: Tasks that run when changed should likely be handlers
    hcloud_roles/hcloud_server/tasks/main.yml:119 Task/Handler: Discord notification

    risky-file-permissions: File permissions unset or incorrect
    roles/address_checker/tasks/main.yml:13 Task/Handler: DB volume directory
\end{lstlisting}

Kattavampi ulostulo on esitelty liitteessä \ref{lst:ansible-lint-example}. Liitteestä
nähdään kuinka kehittäjältä jää helposti useita yksinkertaisia ongelmia huomaamatta ja
saman ongelman toistuvan useasti.

\subsection{Automaattinen ajaminen Gitlab CI:n avulla}

Gitlabia tietovarastona (repository) käyttävät ohjelmistoprojektit voivat käyttää
jatkuvan integraation ja kehittämisen suorittamiseen Gitlab CI/CD:ta, joka mahdollistaa
erilaisten tehtävien suorittamisen automaattisesti koodivarantoon tehtyjen muutoksien
perusteella. Gitlab CI/CD perustuu avoimeen lähdekoodiin ja sen ilmainen versio on ollut
jo aikoinaan parhaimpia ja kattavampia. \parencite{alma9911268330505973}

Yleisesti tunnettuja vaihtoehtoisia ratkaisuja ovat muun muassa Github Actions,
Travis CI, CircleCI ja Jenkins. Tässä työssä kuitenkin esitellään toteutus vain
Gitlab CI/CD:llä, sillä muilla toteuttaminen tapahtuu vastaavasti.

Gitlab CI:ssa tarvittavat määritykset laitetaan `.gitlab-ci.yml`-tiedostoon
\parencite{GitlabCICDDocs}. Esimerkissä \ref{lst:gitlab-ci-ansible-lint} esitelty
yksinkertainen vaihtoehto, jolla saadaan ajettua `ansible-lint` automaattisesti
jokaisen muutoksen myötä.

\lstinputlisting[
    basicstyle=\small,
    caption={Esimerkki yksinkertaisesti Gitlab CI määrityksestä Ansible Lint:n ajamiseen automaattisesti},
    label={lst:gitlab-ci-ansible-lint},
]{code/gitlab-ci-ansible-lint.yaml}

Ylläoleva esimerkki on yksinkertainen eikä sisällä esimerkiksi välimuistin
määrittämistä, jottei työkaluja tarvitse jokaisella suorituskerralla asentaa
uusiksi. Lisäksi työkalun ajaminen joka ei kerta ei välttämättä ole suositeltava
ratkaisu mikäli muutokset eivät kohdistuneen mihinkään, joka vaikuttaisi
lopputulokseen.

\subsection{Tulosten näyttäminen Gitlabissa}

Gitlab CI vertailee koodinlaaturaportteja eri haarojen välillä ja näyttää
tulokset Gitlabin käyttöliittymän kautta \parencite{GitlabCICDDocs}. Kuvassa
\ref{fig:gitlab-code-quality-improved} on näytetty esimerkki siitä, kun
koodin laatu paranee muutosten myötä.

\begin{figure}[h!]
    \includegraphics[width=\textwidth]{figures/gitlab-code-quality-improved}
    \caption{Kuvankaappaus Gitlabin Merge Requestin yleiskatsauksesta}
    \label{fig:gitlab-code-quality-improved}
\end{figure}

On hyvä huomioida, että kohdehaarassa tulee olla muodostettu vastaava raportti
vähintään kerran, jotta vertailua voidaan suorittaa. Muodostetut raportin on
mahdollista ladata Gitlabin käyttöliittymän kautta.\parencite{GitlabCICDDocs}
