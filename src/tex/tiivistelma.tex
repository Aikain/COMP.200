Työssä perehdytään tutkimuksissa havattuihin virheisiin, joita syntyy kun
käytetään Ansiblea keinoa toteuttaa infrastruktuuri. Havatuista virheistä
esitellään muutama yleisin, ja niihin mahdollisia ratkaisuja.

Tarkastellut virheet voidaan jakaa kahteen eri kategoriaan;
tietoturvahajuihin ja huonoihin käytäntöihin. Tietoturvahajut ovat
mahdollisia ongelmia tietoturvaan liittyen, kun puolestaan huonot
käytännöt ovat vain määritelty huonoksi tavaksi toteuttaa asioita.

Tietoturvahajuilla löytyneitä virheitä ovat kovakoodatut salaisuudet
ja eheystarkistuksen puuttuminen. Huonoja käytänteitä puolestaan
riittämätön modulaarisuus ja idempotensin rikkominen. Jokaiseen
virheistä on olemassa ratkaisuja.

Työn lopuksi ahdollisena työkaluna virheiden havaitsemiseen esitellään
ansible-lint-ohjelma.