Tässä työssä perehdytään eri tutkimuksissa ilmenneisiin virheisiin, joita
esiintyy, kun käytetään Ansiblea keinona toteuttaa infrastruktuuri.
Ilmenneistä virheistä esitellään muutama eniten esille tulleista, jotta
saadaan käsitys minkälaisia virheet voivat olla. Virheisiin esitetään
myös ratkaisuja, joita käyttämällä vältyttäisiin kyseisiltä virheiltä.

Tarkastellut virheet voidaan jakaa kahteen eri kategoriaan; tietoturvahajuihin
ja huonoihin käytäntöihin. Tietoturvahajut ovat mahdollisia virheitä, jotka
mahdollisesti heikentävät toteutuksen tietoturvaa. Huonot käytännöt puolestaan
ovat yleisesti määritelty huonoksi tavaksi toteuttaa kyseinen asia. Näistä
selvästi kriittisempiä ovatkin juuri tietoturvahajut, jotka voivat aiheuttaa
merkittäviä ongelmia, kun huonot käytänteet puolestaan ovat enemmän
kehittämisen ja ylläpidettävyyden kannalta häiritseviä.

Tietoturvahajuissa selvästi eniten esille on tullut salaisuuksia
kovakoodaaminen. Dokumentaatiossa lähes poikkeuksetta esitetään aina
kovakoodatuilla salaisuuksilla, jotta dokumentaatio saadaan pidettyä
mahdollisimman yksinkertaisena ja selvänä. Toinen useasti esille noussut
tietoturvahaju oli eheystarkistuksen puuttuminen.

Huonoissa käytänteissä esille nousivat riittämätön modulaarisuus ja
idempotenssin rikkominen. Riittämätön modulaarisuus ongelma tulee vastaan
sovelluksen laajentuessa, eikä sitä ole aluksi välttämättä helppo havaita.
Projektin lähtiessä huonosti liikkeelle voi olla myöhemmin hankala lähteä
korjaamaan. Idempotenssin rikkomisella voidaan aiheuttaa ympäristöihin
yllättäviä ongelmia, joita ei välttämättä pystytä toistamaan ja sen myötä
korjaaminen voi olla haastavaa.

Esitellyt virheet ovat vain pieni osa pienestä joukosta tutkimuksia eikä
kaikkien löydettyjen ongelmien käsittely ole mitenkään mielekästä. Ei ole
myöskään oletettavaa, että kukaan pystyisi tiedostamaan ja hallitsemaan
kaikkia mahdollisia ongelmia. Automaattiset työkalut virheiden tunnistamiseen
eivät ole vielä kattavimpia, mutta joitakin ongelmia on mahdollista tunnistaa.
Yhtenä vaihtoehtoja työn loppupuolella on esitelty ansible-lint-ohjelma, joka
kykenee huomauttamaan joistakin huonoista käytänteistä.
