Teknologian kehityksen myötä palvelinympäristöjen vaatimukset ovat muuttuneet merkittävästi.
Nykyisin suosittuja palveluita on aikoinaan voitu ylläpitää henkilökohtaisilla koneilla.
Kuitenkaan nykyään käyttäjämäärien kasvun, ylläpidollisten vaatimusten ja monien muiden
syiden vuoksi ei vain jotenkuten toimimaan saatu ratkaisu ole kelvollinen. Eri vaihtoehtoja on
nykyään lukemattomia. Tässä työssä on tarkoituksena tutustua yhteen näistä vaihtoehdoista
ja tarkastella joitakin käyttäjien aiheuttamia yleisimpiä ongelmia sen käytössä, joita muissa
tutkimuksissa on havaittu.

Yhtä palvelinympäristöjen hallintaan tarkoitettua keinoa kutsutaan nimellä "infrastruktuuri
koodina" (IaC, Infrastructure as Code). IaC:lla nimensä mukaisesti infrastukruuri rakennetaan
koodin avulla. Jokainen yksittäinen toiminta on määritelty koodiin ja tämän jälkeen koodi
suoritetaan, jotta halutut muutokset saadaan tehtyä. Yksi tällaista keinoa käyttävistä
ratkaisusta on Ansible. Vaihtoehtoisia ratkaisuja ovat muun muassa Puppet, Salt ja Chef.
Tässä työssä kuitenkin keskitytään nimenomaan Ansiblella toteutettuun ratkaisuun.

Ensimmäisessä luvussa käydään läpi Ansiblen perusteet, jotka ovat merkittäviä Ansiblen
käyttämisen kannalta. Luvussa käydään läpi perusteita, kuten muuttujia ja käsittelijöitä,
sekä käydään läpi, miten Ansible rakentuu aina pienistä yksittäisistä tehtävistä kokonaisiksi
playbookeiksi. Työn taustalta löytyy useampi ympäristö, joka on toteutettu Ansiblen avulla,
joten työssä on yksittäisiä esimerkkejä näistä ympäristöistä. Työssä ei kuitenkaan työn
luonteen vuoksi käsitellä konkreettista toteutusta vaan esitellään vain yleisellä tasolla,
miten Ansible toimii.

Seuraavassa luvussa tuodaan esille joitakin tutkijoiden havaitsemia virheitä, joita käyttäjät
voivat mahdollisesti tehdä Ansiblea käyttäessä. Esitellyt virheet ovat tulleet esille useita
kertoja tutkimuksissa. Virheiden esittelyn lisäksi näytetään joitakin mahdollisia ratkaisuja,
joilla voidaan välttää kyseisiä virheitä. Viimeisessä luvussa käydään läpi yksi mahdollinen
ratkaisu virheiden automaattiseen tunnistamiseen.
