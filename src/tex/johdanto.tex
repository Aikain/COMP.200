Teknologian kehityksen myötä palvelinympäristöjen vaatimukset ovat muuttuneet merkittävästi.
Tänä päivänä suosittuja palveluita on aikoinaan voitu ylläpitää henkilökohtaisilla koneilla.
Kuitenkaan tänä päivänä käyttäjämäärien kasvun, ylläpidollisten vaatimusten ja monien muiden
syiden vuoksi ei vain jotenkin toimimaan saatu ratkaisu ole kelvollinen. Eri vaihtoehtoja on
tänä päivänä lukemattomia. Tässä työssä on tarkoituksena esitellä yksi näistä vaihtoehdoista
ja tarkastella joitakin yleisimpiä käyttäjien aiheuttamia ongelmia sen käytössä, joita muissa
tutkimuksissa on havaittu.

Yhtä palvelinympäristöjen hallintaan tarkoitettu keinoa kutsutaan nimellä "infrastruktuuri
koodina" (Infrastructure as Code). Yksi tällaista keinoa käyttävistä ratkaisusta on Ansible.
Vaihtoehtoisia ratkaisuja ovat mm. Puppet, Salt ja Chef. Tässä työssä kuitenkin keskitytään
nimenomaan Ansiblella toteutettuun ratkaisuun. Ensimmäisessä luvussa avataan hiukan tarkemmin
millaista on tehdä infrastruktuuri koodina ja mitä hyötyä sillä saavutetaan. Samalla
esitellään joitakin erovaisuuksia eri ratkaisujen välillä.

Seuraavassa luvussa käydään läpi Ansiblen perusteet, jotka ovat merkittäviä Ansiblen
käyttämisen kannalta. Luvussa käydään läpi perusteita, kuiten muuttujia ja käsittelijöitä
sekä käydään läpi miten Ansible rakentuu aina pienistä yksittäisistä tehtävistä kokonaisiksi
reseptikirjoiksi. Työn taustalta löytyy useampi ympäristö, joka on toteutettu Ansiblen avulla,
joten  työssä on yksittäisiä esimerkkejä näistä ympäristöistä. Työssä ei kuitenkaan työn
luonteen vuoksi käsitellä konkreettista toteutusta vaain esitellään vain yleisellä tasolla,
miten Ansible toimii.

Viimeisessä luvussa tuodaan esille joitakin tutkijoiden havaitsemia virheitä, joita käyttäjät
voivat mahdollisesti tehdä Ansiblea käyttäessä. Valitut virheet painottavat erityisesti
havaittuihin tietoturvahajuihin. Lisäksi avataan joitakin mahdollisia ratkaisuja, joilla
voidaan välttää kyseisiä virheitä.

Test \parencite{AnsibleDocs,alma9911217298805973,KiefMorris2020IaC2,RahmanAkond2021SSiA,JamesFreeman2020HEAo,GauravAgarwal2021MDP,JamesFreeman2020PA2}.

asdf